% Chapter 1

\chapter{Descripción del proyecto} % Main chapter title

\label{Chapter1} % Change X to a consecutive number; for referencing this chapter elsewhere, use \ref{ChapterX}

%----------------------------------------------------------------------------------------
%	SECTION 1
%----------------------------------------------------------------------------------------

\section{Resumen del proyecto}
Lorem ipsum dolor sit amet, consectetur adipiscing elit. Aliquam ultricies lacinia euismod. Nam tempus risus in dolor rhoncus in interdum enim tincidunt. Donec vel nunc neque. In condimentum ullamcorper quam non consequat. Fusce sagittis tempor feugiat. Fusce magna erat, molestie eu convallis ut, tempus sed arcu. Quisque molestie, ante a tincidunt ullamcorper, sapien enim dignissim lacus, in semper nibh erat lobortis purus. Integer dapibus ligula ac risus convallis pellentesque.
\\
\\
Lorem ipsum dolor sit amet, consectetur adipiscing elit. Aliquam ultricies lacinia euismod. Nam tempus risus in dolor rhoncus in interdum enim tincidunt. Donec vel nunc neque. In condimentum ullamcorper quam non consequat. Fusce sagittis tempor feugiat. Fusce magna erat, molestie eu convallis ut, tempus sed arcu. Quisque molestie, ante a tincidunt ullamcorper, sapien enim dignissim lacus, in semper nibh erat lobortis purus. Integer dapibus ligula ac risus convallis pellentesque.
\\
Lorem ipsum dolor sit amet, consectetur adipiscing elit. Aliquam ultricies lacinia euismod. Nam tempus risus in dolor rhoncus in interdum enim tincidunt. Donec vel nunc neque. In condimentum ullamcorper quam non consequat. Fusce sagittis tempor feugiat. Fusce magna erat, molestie eu convallis ut, tempus sed arcu. Quisque
\section{Planteamiento del problema}


\subsection{Planteamiento}

A continuación mencionaré un conjunto de necesidades planteadas por la ley antitrámites del 2012 y que en consecuencia debería cumplirse por los entes de notariado.
\\
Artículo 9 parágrafo A indica que si el solicitante de un trámite tiene papeles que debe presentar en otra entidad puede indicar en que entidad se encuentra tal documento y por lo tanto entre estas entidades deben transferirse estos documentos sin perjucio del mismo trámite. Artículo 15 las entidades públicas o privados que ejerzan funciones pueden tener acceso a los registros públicos esto con el fin de obviar estos certificados. Artículo 19 identificación de personas.
\\
Adicionalmente se puede decir que los trámites propios en las notarias son complejos, muy costosos al momento de obtener copias de un documento y toman tiempo en efectuarse pero además todos estos registros quedan almacenados por cada notaría sin ser compartidos entre ellas u otras entidades que puedan requerir la información allí alojada de una forma rápida. 

\subsection{Formulación}
El prototipo del modelo planteado muestra que es posible mejorar distribución de documentos entre entidades  y hacer que el proceso de validación sea más efectivo?

\section{Estado del arte}
Blockchain es una tecnología reciente y revolucionaria donde se establece una nueva arquitectura\citep{iansiti2017truth}, esto es que, se basa en la confianza de los nodos de la red, plantea eliminar a los terceros o intermediarios que hacen las validaciones y generan la confianza necesaria entre los dos participantes de la transacción, por lo tanto existe una aprobación general en la red frente a una transacción  que puede ser verificada en cualquier momento en el pasado o el presente \citep{crosby2016blockchain}
\\
Blockchain se comporta como un libro de transacciones, basado en cifrado lo cual garantiza la transparencia y seguridad en cada transacción, sin ahondar técnicamente en su funcionamiento podemos indicar que cada transaccion es inalterable, aunque de fondo lo es, podría ser dectectado el fraude con facilidad y descartando la cadena en cuestión, por lo tanto en un símil un bloque, con un conjunto de transacciones, puede ser alterado pero del mismo modo podrá ser dectectado y descartado por los nodos honestos de la red \citep{nakamoto2009bitcoin}
\\
Y si bien durante este proceso hemos mencionado que es una conjunto de registros distribuido y que gracias al cifrado podemos garantizar la transparencia en las transacciones, también, se pueden anonimizar las transacciones ya que no es necesario saber quien la realiza sino solo su identificador público (clave pública) \citep{crosby2016blockchain}, basado en la tecnologia de cifrado publico/privado garantizamos que los registros son irrefutables \citep{banafa2017Blockchain} y que de por si garantiza la comunicación entre las partes \citep{iansiti2017truth}


%\subsection{Blockchain como protocolo}
\subsection{Contratos inteligentes}
Son basicamente un conjunto de reglas programadas que ejecutan los terminos de un contrato de forma automatica al cumplirse o no estas condiciones \citep{crosby2016blockchain}
Basados en que Blockchain  tiene control de algunas variables como tiempo \citep{kosba2016hawk} y los participantes de una transacción es practicamente un trabajo adicional que se apalanquen los contratos a una tecnología implementada con Blockchain que permitirá verificar controlar y ejecutar con mayor facilidad.

\subsection{Propiedades Inteligentes}
``Es otro concepto relacionado al control de un activo/bien/propiedad mediante los contratos inteligentes "
\citep{crosby2016blockchain}

\subsection{Monedas colereadas}
Si respresentamos los objetos existentes en una transacción con una etiqueta (colorear el objeto) para marcar a ese objeto como si fuera en realidad un representación del mundo real (activo/bien/propiedad), eje. unas acciones.
De esta forma se puede almacenar los movimientos de estos en las transacciones pero idetificando claramente a cada una de estas etiquetas.
\\
Se podría poner la propiedad de un auto o una casa en una transacción y moverla de un propietario a otro. \citep{crosby2016blockchain}



\subsection{Aplicaciones}

\subsubsection{Valores privados}
Las bolsas de valores listan acciones de la compañía en un mercado secundario para funcionar de forma segura con operaciones de liquidación y compensación de manera oportuna, ahora es posible para las empresas que emitan directamente las acciones a través de Blockchain. Estas acciones puede ser compradas y vendidas en un mercado que se encuentra en la cadena de bloques. \citep{crosby2016blockchain}

\subsubsection{Notariado público}
Gracias a las características de Blockchain las cuales garantizan que las transacciones son firmadas por el creador y receptor, que cada transaccción se registre con una marca de tiempo y se mantenga la transparencia e integridad son garantias que un documento  tendrá un creador y que este documento fue creado en un momento de tiempo y que el registro de ese documento se mantendrá en el tiempo inmutable. \citep{zheng2016blockchain}
\\Podemos adicionar que esto elimina la necesidad de que un tercero valide alguna de las caracterisitcas previas sino que de la misma forma este documento estará distribuido a lo largo de la red lo cual generará que los costos del proceso disminuyan \citep{crosby2016blockchain}
\\
Para el 2018 en Colombia el gobierno está apalancando en esta tecnología para su proyecto de restitución de tierras \footnote{https://www.elespectador.com/economia/asi-se-utiliza-blockchain-para-garantizar-la-restitucion-de-tierras-articulo-809025} a personas victimas del conflicto armado.
\subsubsection{Propiedad intelectual}
Si bien cualquier recurso digital se le puede aplicar el modelo de blockchain, a medio electronicos como películas, canciones y demás que involucren propiedad intelectual les impacta de buena forma está tecnología ya que con esta tecnología será garante de que no se puedan generar duplicaciones no autorizadas. \citep{huckle2016internet}
\\
''Aquí es donde el blockchain puede jugar un papel. La tecnología puede ayudar a mantener una gran base de datos distribuida precisa de la propiedad de los derechos musicales información en un libro público. Adicionalmente a la información de propiedad de derechos, la división de regalías para cada trabajo, según lo determinado por Smart Contracts, podría ser agregado a la base de datos. Esta Los contratos inteligentes a su vez definirían las relaciones de relación entre diferentes partes interesadas  y automatizar sus interacciones'' \citep{crosby2016blockchain}

\subsubsection{Intenet de las cosas}
Internet de las cosas (IoT por sus siglas en ingles) es una tencnología emergente y que con seguridad no ha logrado su máximo de madurez, al igual que Blockchain, IoT tiene como finalidad integrar los elemento de nuestro diario vivir con nosotros de formas autonomas y transparentes.
\\
En el comercio electrónico se esta proponiendo un nuevo modelo basado en Blockchain y contratos inteligentes de propiedades inteligentes, la idea es que las personas reciban transacciones al cumplirse una condición  a partir de señales de sensores emitidas por los  objetos inteligentes. \citep{zheng2016blockchain}
\\
Para añadir otro elemento Blockchain puede proveer una descentralizada red que habilita a los  objetos inteligentes a interactuar con criptomonedas y garantizar que todas sus interacciones estan completamente validadas por la red\citep{crosby2016blockchain}, de esta forma los usuarios podrán tener tranquilidad que ningun dispositivo podrá ser vulnerado o manipulado en beneficio o en contra de un usuario.
\\
Un ejemplo con una tecnología emergente hace un par de años y que ahora parece consolidarse un poco es Uber, está podría implementar un modelo de IoT y Blockchain; para cuando un pasajero llega a su destino el cobro se implemente de inmediato mediante contratos inteligentes, pero de la misma forma cuando el conductor de Uber no garantiza el servicio el pasajero se puede ver beneficiado cobrando una multa por un mal servicio o daño en la reputación por alguna circustancia.\citep{huckle2016internet}


\subsubsection{Cuidado de la salud}
Uno de los campos de acción de Blockchain es permitir garantizar la identidad de los pacientes, un paciente debidamente carnetizado y enrolado permite que toda su información sea el unico que puede acceder a ella y permitir a quienes les da acceso para que sea consultada. \citep{angraal2017blockchain} Pero sin lugar a dudas lo más beneficioso para un paciente es que puede aprovechar la ventaja de que su historia está distribuida y como se beneficia de esto, supongamos un paciente crónico alérgico a una gran cantidad de medicamentos y/o compuestos químicos tantos como para no poder mencionarlos todos o un paciente con una gran cantidad de cirugias a lo largo de su vida, como hacen ahora para poder cambiar de ubicación? pues deben llevar su historia médica en papeles, Blockchain llega al rescate todo estará disponible para ser consultado cualquier médico autorizado podrá revisarlo con los conceptos de sus colegas no solo lo que recuerda el paciente.
\\
Esto nos lleva a otro uso y es que entre aplicaciones que usan la tecnología se podría compartir información de un paciente, podrían compartir autorizaciones, permisos y acuerdos firmados, esto sería de gran ayuda para reducir los tiempos de procesos entre organizaciones que prestan servicios de salud. \citep{angraal2017blockchain}

\subsection{Retos de blockchain}
\subsubsection{Regulación}
El tema regulatorio es bastante discutible, la tecnología siempre avanza mucho mas rápido que los gobiernos, en especial latino americano, como es de esperarse Blockchain se está asentando y el gobierno aun no ve la necesidad de hacerlo, lo difícil de esta postura es que cuando se vea en la necesidad puede que sea muy restrictivo impactando de forma negativa e impidiendo todo su potencial\citep{banafa2017Blockchain}, permitiendo que solo algunos se benficien de este mismo como menciona \citep{cabral2018EstadoArte} deberá ser equilibrado y regular lo suficiente para impedir que se use de forma fraudulenta y también proteger al más debil. 
\subsubsection{Escalabilidad}
El constante crecimiento de la base de datos, con un tamaño 100,18 GB \footnote{cifras 2016}, todas las transacciones deben ser almacenadas para poder validar cada transacción, además que por definición el tiempo entre bloques de transacciones tiene un retraso de tiempo lo que lleva a que se procesen solo 7 transacciones por segundo lo que lleva a que en un mundo donde se implemente esta tencnología impediría que se use en tiempo real \citep{zheng2016blockchain}
\\Además si un usuario es nuevo en el ecosistema y pretende hacer una transacción debera primero sincornizar su base de datos descargando toda la cadena de bloques y validar las transacciones antes de poder realizar su transacción lo cual le tomará un tiempo en ejecutar \citep{crosby2016blockchain}
\subsubsection{Resistencia al cambio}
Como en todo proyecto de tecnología, por lo general innovadores, se debe hacer la gestión del cambio para no impedir que los actores generen resitencia y el proyecto fracase en una organización, en este caso puede que no sea una organización pero si el público en general puede ser renuente al uso, debido a que se pueden generar mitos sobre el uso de la tecnología como problemas de seguridad, ineficiencia, lentidtud, etc. En la actualidad los intermediarios (eje. Visa, masteercard, Uber)\citep{crosby2016blockchain} brindan la seguridad que ningun problema se pueda presentar (aun cuando se presentan) y son garantes de que las transacciones se cumplan con satisfacción de las partes. 
\subsubsection{Integración con el pasado}
Uno de los grandes problemas que se pueden presentar en la aplicación de la tenoclogía es la historia existente, primero el problema que conlleva la migración de esta historia implicaría tiempo y costos altos\citep{crosby2016blockchain} y ademas ya que de alguna forma se debe integrar pero toda esta historia se registrarían como nuevas transacciones en la cadena y se generaría confusiones y si no se tiene cuidado el orden de documentos, por ejemplo podría alterarse cronologicamente si no se hace un uso adecuado, aunque este reto propiamente afecta a la organización que quiera aplicar está tecnología deberá tener en cuenta un proyecto alterno de  gestión del conocimiento por ejemplo, para mitigar este riesgo.

\subsubsection{Fraude}
Debido a la naturaleza de Blockchain se pueden presentar intentos de fraude, pero como se indica en el artículo originar de Blockchain se espera que el esfuerzo tan alto de hacer fraude se vea mejor recompensado por el hecho de hacer el esfuerzo por mantener un cadena de bloques honesta\citep{nakamoto2009bitcoin}, pero en algún punto el camino tenderá a torcerse y deberá mantenerse controles y regulaciones para controlar estos intentos de una forma certera, encaminados con regulaciones de ley para que la comunidad se sienta protegida

\subsubsection{Súper computadoras}
La capacidad de computo es una de las piedras angulares de Blockchain y la prueba de trabajo es un control de varios elementos del protocolo Blockchain, por lo tanto si llegará a existir alguna super computadora que siempre estuviera en la capacidad de generar primero la prueba de esfuerzo implicaría que la dificultad \citep{nakamoto2009bitcoin} deberá aunmentar hasta que los mineros puedan generarlo aleatoriamente, pero por obvias razones los demas mineros continuaran en desventaja ya que su capacidad de computo es menor.
\\
Pero ahora si, como es de esperarse, la capacidad de computo de la comunidad minera comienza a mejorar la difcultad, como se mencionó antes, debe aumentar pero existe una limitante en esa dificultad y es que la cantidad de ceros no puede crecer infinitamente por que si no no quedará espacio para la información lo que obligaría al protocolo a migrar de algoritmo de seguridad con implicaciones e impactos para la red elevados\citep{crosby2016blockchain}


\section{Objetivos del proyecto}

\subsection{Objetivo general}
Proponer una arquitectura empresarial para las oficinas de notariado con el fin de mejorar la comunicacion entre entidades mediante la implementación de la tecnología de Blockchain. 

\subsection{Objetivos específicos}
\begin{itemize}

\item Definir un modelo de Arquitectura empresarial para las oficinas de notariado público detectando las necesidades para la adopción de blockchain

\item Elaborar un prototipo basado en la tecnología Blockchain donde se represente el proceso y se demuestre que se puede controlar la distribución de documentos y sus validaciones
\end{itemize}




